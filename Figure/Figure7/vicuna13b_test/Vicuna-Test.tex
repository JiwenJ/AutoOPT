\documentclass{article}%
\usepackage[T1]{fontenc}%
\usepackage[utf8]{inputenc}%
\usepackage{lmodern}%
\usepackage{textcomp}%
\usepackage{lastpage}%
\usepackage{tcolorbox}%
\usepackage{amsmath}%
%
\tcbuselibrary{most}%
%
\begin{document}%
\normalsize%
\newtcolorbox{mybox}[1]{breakable, colback=white, colframe=black!50, fonttitle=\bfseries, title=#1}%
\noindent\textbf{Question 23}: A hiker wants to visit 8 different landmarks and return to his hotel. The landmarks are numbered from 1 to 8 and the distance and difficulty level between each pair of landmarks are given in the following table: 
Landmark	1	2	3	4	5	6	7	8 
1 	-	2	4	6	3	1.5	5	0.5 
2 	-	-	3	5	4	2.5	6	1.5 
3 	-	-	-	4	5	3.5	7	2.5 
4	-	-	-	-	6	4.5	8	3.5 
5	-	-	-	-	-	5.5	9	4.5 
6 	-	-	-	-	-	-	10	5.5 
7 	- 	- 	- 	- 	-	-	-	6.5 
8 	-	-	-	-	-	-	-	- 
The difficulty level between each pair of landmarks is equal to the distance multiplied by a constant factor of $0.2 per mile. The hiker has a limited amount of energy and can only visit each landmark once. The hiker wants to minimize the total difficulty level while visiting each landmark exactly once. How should the hiker plan his route?
\\\\%
\textbf{GPT-4}: This problem is a variant of the well-known Traveling Salesman Problem (TSP), which is a combinatorial optimization problem. In this case, the hiker wants to visit each landmark exactly once, and return to the hotel (which we can consider as a landmark itself), while minimizing the total difficulty level. The difficulty level between landmarks is given as the product of the distance and a constant factor. 

The problem can be formulated as an Integer Programming problem. First, let's define our decision variables, objective function, and constraints.

#### Decision Variables:
Let's denote $x_{ij}$ as the decision variable that equals 1 if the hiker travels from landmark $i$ to landmark $j$, and 0 otherwise. Here, $i,j \in \{1,2,...,8\}$ and $i \neq j$.

#### Objective Function:
The objective of the hiker is to minimize the total difficulty level. The difficulty level between each pair of landmarks is the distance multiplied by a constant factor of 0.2. So, the objective function can be written as:
$$
\min Z = 0.2 \sum_{i=1}^{8} \sum_{j=1,j\neq i}^{8} d_{ij} \cdot x_{ij}
$$
where $d_{ij}$ is the distance from landmark $i$ to landmark $j$.

#### Constraints:
1. Each landmark is visited exactly once:
$$
\sum_{i=1,i\neq j}^{8} x_{ij} = 1, \quad \forall j \in \{1,2,...,8\}
$$
$$
\sum_{j=1,j\neq i}^{8} x_{ij} = 1, \quad \forall i \in \{1,2,...,8\}
$$

2. To eliminate sub-tours (meaning that we can't have a tour that visits a subset of the landmarks and then returns to the original landmark), we introduce additional constraints using auxiliary variables $u_{i}$:
$$
u_{i} - u_{j} + 8 \cdot x_{ij} \leq 8 - 1, \quad \forall 2 \leq i \neq j \leq 8
$$

3. Decision variables and auxiliary variables limitations:
$$
x_{ij} \in \{0,1\}, \quad \forall i,j \in \{1,2,...,8\} \quad \text{and} \quad i \neq j
$$
$$
u_{i} \in \{1,2,...,8\}, \quad \forall i \in \{1,2,...,8\}
$$

This mathematical model is an Integer Programming model due to the binary nature of the decision variables $x_{ij}$, and it captures the essence of the problem described. The model ensures that the hiker visits each landmark exactly once, and then returns to the hotel, while minimizing the total difficulty level. 

Please note that solving this problem to optimality could be computationally expensive, especially as the number of landmarks increases, due to the combinatorial nature of the problem. However, there exist heuristic and approximation algorithms that can provide good solutions for large instances of the problem.\\\\%
\textbf{Vicuna zero-shot}: To model this problem, we can define the decision variables as binary variables x1, x2, ..., x8, where xi=1 if the hiker visits landmark i and xi=0 otherwise. The objective function can be defined as the total difficulty level of the route, which is equal to the sum of the product of the distance between each pair of landmarks and the constant factor of $0.2 per mile, multiplied by the binary variable corresponding to the second landmark in the pair.

Mathematically, the problem can be formulated as follows:

Minimize: ∑i=1 to 8 ∑j=1 to 7 (xi*dij*0.2)

Subject to:

* ∑i=1 to 8 xi = 1 (the hiker must visit all 8 landmarks)
* xi ∈ {0,1} (the hiker can only visit each landmark once)

where dij is the distance between landmarks i and j.

This problem is a mixed integer programming problem, which can be solved using a mixed integer programming solver such as Gurobi or CPLEX. The solution to this problem will be a binary vector x1, x2, ..., x8 that minimizes the total difficulty level while visiting each landmark exactly once.\\\\%
\textbf{Vicuna one-shot}: To model this problem, we can use linear programming. Let's denote the number of visits to each landmark as a binary variable xij, where i is the index of the landmark and j is the number of visits (0 for unvisited and 1 for visited). The objective function is to minimize the total difficulty level while visiting each landmark exactly once. The total difficulty level of visiting landmark i is given by the sum of the difficulty level and the number of visits to landmark i. Therefore, the objective function is:

$\min \sum\_{i=1}^8 \sum\_{j=1}^{k\_i} 0.2jx\_{ij}$

where $k\_i$ is the number of landmarks that can be visited before reaching landmark i.

Constraints:

1. The hiker has a limited amount of energy and can only visit each landmark once, so the sum of visits to each landmark is equal to 1:

$\sum\_{i=1}^8 \sum\_{j=1}^{k\_i} x\_{ij} = 1$

2. The binary variable xij is 0 for unvisited landmarks and 1 for visited landmarks:

$x\_{ij} \in {0,1}$

3. The hiker must visit landmark 1 before any other landmark:

$x\_{11} = 1$

4. For each landmark i, the number of visits to landmark i is equal to the number of visits to all landmarks before landmark i:

$\sum\_{j=1}^{k\_1} x\_{1j} + \sum\_{j=1}^{k\_2} x\_{2j} + \dots + \sum\_{j=1}^{k\_i} x\_{ij} = k\_i$

5. The binary variable xij is 0 for unvisited landmarks and 1 for visited landmarks:

$x\_{ij} \in {0,1}$

In LaTeX form, the mathematical model is as follows:

\begin{aligned}
& \text{minimize} && \sum\_{i=1}^8 \sum\_{j=1}^{\\\\%
\textbf{Vicuna few-shot}: To solve this problem, we can formulate it as a mathematical optimization problem. We can represent the problem as follows:

* Decision variables:
	+ $x\_1, x\_2, \ldots, x\_8$, where $x\_i=1$ if the hiker visits landmark $i$ and $x\_i=0$ otherwise.
* Objective function:
	+ Minimize the total difficulty level of the landmarks visited, which is given by the sum of the difficulty level of each landmark multiplied by the number of miles traveled to reach it:
		- $\frac{1}{2}d(i,j)x\_ix\_j$ where $d(i,j)$ is the distance between landmarks $i$ and $j$, and $x\_i$ and $x\_j$ are binary variables indicating whether the hiker visits landmark $i$ or $j$.
* Constraints:
	+ The hiker can only visit each landmark once, so for each landmark $i$, we have $x\_i \leq 1$.
	+ The hiker starts and ends at his hotel, so we have $x\_1 + x\_8 = 1$.
	+ For each landmark $i\\\\%
\end{document}