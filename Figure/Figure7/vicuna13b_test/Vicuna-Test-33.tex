\documentclass{article}%
\usepackage[T1]{fontenc}%
\usepackage[utf8]{inputenc}%
\usepackage{lmodern}%
\usepackage{textcomp}%
\usepackage{lastpage}%
\usepackage{tcolorbox}%
\usepackage{amsmath}%
%
\tcbuselibrary{most}%
%
\begin{document}%
\normalsize%
\newtcolorbox{mybox}[1]{breakable, colback=white, colframe=black!50, fonttitle=\bfseries, title=#1}%
\noindent\textbf{Question 33}: A company has a fleet of 4 vehicles with a capacity of 50 units each. The company needs to deliver goods to 9 customers with the following demands: 10, 15, 18, 20, 12, 16, 19, 17, and 11 units. The company's depot is located at (0,0) and the customers are located at (20,20), (30,40), (50,30), (60,50), (20,50), (30,60), (50,60), (60,20), and (40,40) respectively. The distance between any two locations is calculated using the Euclidean distance formula. How can the company minimize the total distance traveled by its vehicles while satisfying all the customers' demands and not exceeding the vehicles' capacities?
\\\\%
\textbf{GPT-4}: \\\\%
\textbf{Vicuna zero-shot}: This problem can be modeled as a Vehicle Routing Problem (VRP) with capacity constraints and distance constraints.

Decision Variables:

* x1, x2, x3, x4: the number of trips made by each vehicle
* y1, y2, y3, y4: the number of customers visited by each vehicle

Objective Function:

* Minimize the total distance traveled by all vehicles, given by:

Total Distance = (x1 + x2 + x3 + x4) \* (distance between depot and customer 1) + (y1 + y2 + y3 + y4) \* (distance between any two customers)

Constraints:

1. Capacity Constraints:

For each vehicle, the total demand it carries must not exceed its capacity, given by:

x1 + x2 + x3 + x4 <= 4 \* 50 = 200

2. Distance Constraints:

For each customer, the distance between the customer and the depot must be equal to the sum of the distances between the customer and the customers it is visited by the same vehicle, given by:

distance between customer i and depot = (x1 + x2 + x3 + x4) \* (distance between customer i and customer 1) + (y1 + y2 + y3 + y4) \* (distance between any two customers)

3. Vehicle Constraints:

For each vehicle, the number of trips it makes must be greater than or equal to 1 and less than or equal to the total demand, given by:

x1 >= 1
x1 + x2 + x3 + x4 <= 4 \* 50 = 200

4. Customer Constraints:

For each customer, the number of visits it receives must be greater than or equal to 0 and less than or equal to the demand, given by:

y1 >= 0
y1 + y2 + y3 + y4 <= 9 \* 50 = 450

5. Integrality Constraints:

All decision variables must be integers.

This problem can be solved using a mixed-integer\\\\%
\textbf{Vicuna one-shot}: Let's establish a mathematical model for this problem. We will denote the number of trips made by each vehicle as\\\\%
\textbf{Vicuna few-shot}: To minimize the total distance traveled by the vehicles, we can use the Hungarian method or the assignment problem.

First, we need to create a table where each row represents a vehicle and each column represents a customer. The values in the table represent the number of units that each vehicle can deliver to each customer.

| Vehicle 1 | Vehicle 2 | Vehicle 3 | Vehicle 4 |
| --- | --- | --- | --- |
| 10 | 15 | 18 | 20 |
| 12 | 16 | 19 | 17 |
| 11 | 17 | 18 | 19 |
| 10 | 15 | 18 | 20 |
| 15 | 18 | 20 | 12 |
| 18 | 20 | 12 | 16 |
| 19 | 17 | 18 | 19 |
| 17 | 18 | 19 | 17 |
|\\\\%
\end{document}